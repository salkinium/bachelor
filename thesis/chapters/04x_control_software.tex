\section{Control Software}
\label{sec:control_software}

The motes and temperature boxes are connected via USB-to-Serial converters to a computer, and controlled with a Python program using the TinyOS Python SDK~\cite{tinyos.net}.
Since all motes are connected to one computer, packet loss due to colliding transmissions is avoided via central scheduling of transmissions.
The software holds virtual representations of the mote, temperature controller and link, and logs experiment results to disk.
A simple scripting language allows the definition of commands that are interpreted by the runtime and executed serially so that experiments can be described in independent and compact form.
Multiple scripts can be added, so that experiments can run unsupervised around the clock.

During an experiment all transmissions and receptions are written to a log in ASCII format, which includes a timestamp, sending and receiving node ids, a sequence number, link quality metadata, mote temperature and the entire \ac{MPDU} payload, including \ac{FCS}.%, as shown in Listing~\ref{lst:log_example}.
These logs describe the entire experiment in a unprocessed format, which is then read by the evaluation scripts that reassemble them into messages and links.

This modular software setup allows for a lot of freedom when designing these experiments, as all evalutation data is based upon these unprocessed logs.
This enables us to use one experiment for multiple purposes by focusing on different parameters, of which we will make use later in in Chapter~\ref{chap:forward_error_correction}.

\begin{listing}[t]
\begin{lstlisting}[breaklines=true]
timestamp=2014-05-03 09:22:21,752	mode=tx	id=0	seqnum=0	temperature=29.9	length=93	data=[...]	power=3
timestamp=2014-05-03 09:22:21,753	mode=rx	id=1	seqnum=0	temperature=33.0	timeout=1

timestamp=2014-05-03 09:22:21,890	mode=tx	id=1	seqnum=1	temperature=33.0	length=93	data=[...]	power=3
timestamp=2014-05-03 09:22:21,890	mode=rx	id=0	seqnum=1	temperature=29.9	length=93	data=[...]	rssi=-90	lqi=104 crc=1 timeout=0
\end{lstlisting}
\caption{A log excerpt showing two transmissions, of which the first reception timed out. The raw data field is ommited for clarity.}
\label{lst:log_example}
\end{listing}
