\section{Control Software}

The motes and temperature boxes are connected via USB-to-Serial converters to a computer, and controlled with a Python program using the TinyOS Python SDK~\cite{tinyos.net}.
Since all motes are connected to one computer, packet loss due to colliding transmission was not possible. 

The software hold virtual representations of the mote, temperature controller and link, and logs experiment results to disk.
A simple scripting language allows the definition of commands that are interpreted by the runtime and executed serially so that experiments can be described in independent and compact form.
A short exerpt of an experiment script is shown in Figure~\ref{fig:script_exampe}.
Multiple scripts can be added, so that experiments can run unsupervised around the clock.


\begin{listing}[h]
\begin{lstlisting}[breaklines=true]
set temperature:	box=1	temperature=30
set temperature:	box=0	temperature=30
# wait for 30C in both boxes
wait temperature:	timeout=10000

# alternate sending 2 messages from box 0 and 1
send message:	from=0,1	power=3	data=rs(80, 70, random)	timeout=0.15	period=0	repeat=10000	bursts=2

# set new temperature
set temperature:	box=0	temperature=40
# but start sending immediately
send message:	from=0,1	power=3	data=rs(80, 70, random)	timeout=0.15	period=0	repeat=10000	bursts=2
\end{lstlisting}
\caption{A script exerpt showing sending RS(80, 70) encoded random payload}
\label{lst:script_example}
\end{listing}


