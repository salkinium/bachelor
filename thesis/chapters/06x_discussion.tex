\section{Discussion}

Considering that the goal of using an \ac{FEC} in the context of low power networks such as \ac{WSN}s is to maximize energy efficiency of communications, we are not only looking for the $k$ with maximum throughput, but also with the least retransmissions.
We also have to consider that the link quality can already be poor at low temperature, therefore using no parity bytes at low temperatures is not a good idea either.

As per our findings, we propose one possible \ac{RS} scheme starting with $k=70$, which is 12.5\% overhead (the same as the popular $RS(255,223)$~\cite{Ma2009}), at low temperatures and linearly increase this to $k=60$ (or 25\% overhead) as receiver temperature increases to $70\,^{\circ}\mathrm{C}$ and above.
Starting with $k=70$ will give some protection against a random decrease in link quality as seen in Figure~\ref{fig:throughput_link_01_receiver_fec}.
In cases of extremely high bit error, such as in Figure~\ref{fig:throughput_link_10_receiver_fec}, we can still regain some throughput by using $60-80\%$ coding overhead with $k=30$ or $k=20$.
However, considering the extremely low \ac{PRR} in this area (compare with Figure~\ref{fig:prr_link_10_receiver_fec}), most of these messages will likely never be received anyway, which makes this option only viable when communication at these temperatures is absolutely necessary.

The results of Boano~\etal~\cite{Boano2013} strongly suggested that the loss in \ac{PRR} is more pronounced when heating the transmitter than the receiver.
This would have allowed us to use local temperature measurements to adapt the coding strength of our \ac{FEC} scheme to counteract the loss in \ac{PRR}.
Unfortunately, our results firmly oppose the findings of Boano~\etal{}: heating the receiver creates a higher loss in \ac{PRR} than heating the transmitter, therefore making such an approach unusable.
However, using temperature as another source for assessing link quality has the distinct advantage of being always available on the receiver locally, compared to \ac{LQI} and \ac{RSSI}, which requires a message to be received first.

We therefore propose another solution to adapt \ac{FEC} strength:
the receiver should monitor its local temperature and send out a warning broadcast to all potential transmitters, before its temperature becomes too high.
The transmitters can then address this mote with the appropriate \ac{FEC} strength.
The advantage of this active, preemptive approach over backchanneling link quality information is that in setups where transmissions only occur sparsely, a ``test'' transmission to judge link quality is not required.

