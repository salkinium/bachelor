\section{Temperature Box}
\label{sec:temperature_box}

Even though previous setups used an infrared lamp~\cite{Boano2013, Hermans2013} to heat the motes directly, we chose to control mote temperature sorely using air temperature.
The mote is placed in a closed styrofoam box and the confined air is heated to the requested temperature and circulated to transfer this heat to the mote.

\subsection{Hardware}
A microcontroller evaluates temperature sensors within the box and locally controls the duty cycles of a small fan and 150W ceramic heating element using a \acs{PID} loop to achieve the requested air temperature.
The microcontroller can communicate over a serial connection to receive the desired temperature and send the current temperature.

Additionally, air temperature is displayed as a hue from blue (cold) via green (warm) to red (hot) on a RGB LED.
The duty cycles of the heating element and fan are mapped to two white LEDs, to provide an immediate status overview and prevent burn injuries.

The box is powered by a 300W PC power supply, which was chosen for its ability to provide 12V, 5V and 3.3V at the required currents, which removes the need for additional (costly) voltage conversion.
A custom designed PCB distributes power from the \acs{ATX} connectors to two high and two medium power \acs{MOSFET} switches and up to five temperature sensors, all controlled by an ATmega328p.

The PCB, heating element and fan are fixed on a wooden baseplate using fasteners and screws.
All custom made parts were prototyped using the PCB mill and laser cutter of the RWTH FabLab~\cite{fablab}.
The styrofoam box has the outside dimensions of $35 \times 35 \times 30$cm with a wall thickness of 5cm which results in a holding capacity of $12.5\ell$.

Figure~\ref{fig:box_hardware_picture} shows the baseplate with controller, heating element, fan and mote harness as labeled.

\begin{figure}[hb]
	\centering
    \includegraphics[width=1\columnwidth]{figures/temperature_box}
	\caption{Baseplate with temperature controller, heating element, fan and sensor mote.}
    \label{fig:box_hardware_picture}
\end{figure}


\subsection{Embedded Software}

The embedded software is written using the xpcc microcontroller framework~\cite{xpcc.io} and implemented as a set of asynchronous control tasks for input parsing and output formatting, temperature sensor evaluation, \acs{PID} loop update and duty cycle generation.
The choice of an ATmega328 as a microcontroller was deliberate as to enable compatibility with the Arduino framework, if this is the preference.

The controller is programmable via \ac{ISP} and will then periodically send the values of all attached temperature sensors, as well as the current heating element power setting, both in human-readable ASCII format.
To set a temperature
All switching frequencies were kept well below 1kHz to avoid any kind of interference with the 2.4GHz band.


\subsection{Performance}

The heating element has enough power to heat the air inside the box up to $120\,^{\circ}\mathrm{C}$.
This can however damage both the styrofoam material as well as the mote, so a hard limit of $90\,^{\circ}\mathrm{C}$ is imposed during experiments.

As seen in Figure~\ref{fig:box_heating_cooling} it takes about 40 minutes to heat up to $90\,^{\circ}\mathrm{C}$ and about 2 hours to cool back down to $30\,^{\circ}\mathrm{C}$.
The \acs{PID} loop is deliberately dampened so that no overshoot in air temperature occurs, however, since the mote temperature naturally lags behind, a slight overshoot in air temperature might actually help achieve desired mote temperature quicker.
However, during the experiments a more granular approach is used similar to Figure~\ref{fig:box_heating_step}.

With a room temperature of $20-25\,^{\circ}\mathrm{C}$, the minimum experiment temperature was set at $30\,^{\circ}\mathrm{C}$ so that it can be reached within reasonable time.
The boxes can be retrofitted with an active piezoelectric cooling element and a fan by connecting them to the two unused \acs{MOSFET} switches on the controller and updating the software, which also allow reaching temperatures below room temperature.

\begin{figure}[H]
	\subfigure[Heating to $90\,^{\circ}\mathrm{C}$, then cooling to $30\,^{\circ}\mathrm{C}$.] {
    	\includegraphics[width=0.5\columnwidth]{figures/box_heating_cooling}
    	\label{fig:box_heating_cooling}
    }
    \subfigure[Temperature increments of $5\,^{\circ}\mathrm{C}$.] {
	    \includegraphics[width=0.5\columnwidth]{figures/box_heating_step}
	    \label{fig:box_heating_step}
	}
	\caption{Typical performance of the boxes over time. Notice the mote temperature (red line) lagging behind air temperature (green line).}
	\label{fig:box_heating_curves}
\end{figure}

